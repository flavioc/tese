A tabulação é um método de resolução particularmente bem sucedido que resolve algumas limitações
do método de avaliação SLD encontrado em sistemas Prolog, tais como a recursão e computação redundantes.
Na tabulação, as primeiras chamadas a subgolos são avaliadas normalmente através da avalição do código do
programa, enquanto que \emph{chamadas similares} são avaliadas através do consumo de respostas guardadas
numa tabela pelo subgolo similar correspondente.
Em geral, podemos distinguir entre duas maneiras para determinar se o subgolo $A$ é similar ao subgolo $B$:
\emph{tabulação por variantes} e \emph{tabulação por subsumpção}.
Na tabulação por variantes, $A$ é similar a $B$ quando eles são iguais por renomeação das variáveis.
Na tabulação por subsumpção, $A$ é similar a $B$ quando $A$ é mais específico que $B$ (ou $B$ é mais geral que $A$).
Isto surge através de um simples princípio: quando $A$ é mais específico que $B$ e $S_A$ e $S_B$ são os respectivos
conjuntos de rrespostas, então $S_A \subseteq S_B$.
Embora a tabulação por subsumpção consegue atingir maiores ganhos em termos de tempo de execução, devido
a aumentar o nível de partilha de respostas, implementar os mecanismos necessários de forma eficiente é bastante
mais difícil em comparação com tabulação por variantes, o que faz com que este último seja muito mais popular
nos motores de tabulação disponíveis.

Esta tese descreve o desenho e integração do mecanismo do Time-Stamped Tries do XSB Prolog no motor de tabulação
YapTab. Os resultados obtidos mostram que os nossos esforços de integração foram bem sucedidos, com \emph{speedups}
comparavéis entre tabulação por variantes e tabulação por subsumpção.

Na segunda parte desta tese apresenta-se o desenho e avaliação de uma nova extensão de baseada na tabulação
por subsumpção chamada Tabulação por Subsumpção Retroactiva (TSR). Esta resolve algumas limitações da tradicional
tabulação por subsumpção, nomeadamente, o facto de a ordem da chamada dos subgolos poderem afectar o seu sucesso e
aplicação. A TSR permite uma partilha completa e bidireccional de respostas entre subgolos, independentemente da
sua ordem de chamada através do corte da avaliação dos golos mais específicos.
Os nossos resultados mostram ganhos consideráveis para programas que podem tomar vantagem do novo mecanismo,
enquanto que programas que não podem beneficiar mostram um overhead mínimo em relação à tabulação por subsumpção
normal.