
In this chapter we summarize the work developed in this thesis. First, we list the various contributions
present in the thesis, next we suggest problems for future research, and then we finish with a final
remark.

\section{Main Contributions}

We can identify two main contributions of this thesis. The first, is the integration and design of a call
subsumption engine into YapTab, originally a variant engine. The second main contribution is the design and implementation
of a new tabling execution mechanism called Retroactive Call Subsumption, that maximizes the sharing of answers
between subsumed/subsuming subgoals.
In order to build these two systems we used the Time-Stamped Tries mechanisms from XSB to implement the
subsumption algorithms and data structures. This process made us understand the differences between the
YapTab and the SLG-WAM tabling engines.

We next detail the most important aspects of the work developed during this thesis:

\begin{description}
   \item[Subsumption-based tabling engine.] Tabling evaluation by using subsumptive checks is now supported in Yap Prolog.
   While we integrated the TST algorithms and data structures from SLG-WAM into YapTab, the modifications made to
   to YapTab were minimal and show that it is possible to add support for subsumption-based tabling to a delaying-based
   tabling engine that already supports variant checks by preserving its own tabling operations and main algorithms.
   In addition, YapTab is also able to mix variant and subsumption-based tabling in the same program by defining
   the evaluation model for each predicate.
   
   \item[Mechanisms to control retroactive-based execution.] This thesis innovates by presenting a novel execution model
   that is able to prune the execution of specific subgoals when a more general subgoal appears. We developed rules
   for pruning a range of choice points and presented the issues that arise and can lead to completion problems when
   transforming consumer nodes into generators. We also developed an efficient mechanism to detect if a subgoal is
   internal to the execution of another subgoal by building a subgoal dependency tree.
   
   \item[Algorithm to find subsumed subgoals in the table space.] We designed a novel and efficient algorithm that
   can detect instances of a subgoal that are currently being evaluated. Our design takes advantage of the existing
   WAM machinery and data areas. In order to prune the search space, we extended the subgoal trie data structure with
   information about the status of the subgoals under a subgoal trie node.
   
   \item[Single Time Stamped Trie table space organization.] In the STST table space organization we have an answer
   trie for each predicate. The answer is itself an time stamped trie and permits greater reuse of answers between
   the subgoals of the same predicate, as answers are represented only once. The design facilitates the pruning of
   subsumed subgoals because the subgoals can easily identify which answers have already been consumed or generated.
   We also designed a new optimization, where we throw away the subgoal trie when the most general subgoal completes,
   therefore saving memory.
   
   \item[Enhanced handling of incomplete subgoals.] The STST table space organization also permits that a certain
   subgoal is able to immediately reuse answers already on the answer trie added by other subgoals. That means that if a
   few subsumed subgoals execute and then a more general subgoal appears we can first load the answers already stored and
   then execute the predicate clauses. This is a very efficient way of handling incomplete tables, because we are
   not only restricted to an incomplete or complete answer set from a single subgoal but we are also able to use
   answers from other subgoals, which is useful if the subgoal stops execution by the cut operator.
   
   \item[Support for mixed tabling checks.] Our final system is able to mix variant, subsumption and retroactive-based
   tabling in the same program. This enables the programmer to choose the best evaluation strategy per predicate,
   which we think greatly augments the power of tabling to be used in real world programming.
   
   \item[Performance Evaluation.] We evaluated the performance of our subsumption-based tabling engine
   against SLG-WAM with very comparable results, which validates our integration efforts. We also verified
   that using call subsumption can potentially cut down on execution time and waste less memory.
   For retroactive-based, we evaluated the overhead of using the new control mechanism with programs that do
   not benefit from retroactive evaluation. For programs were retroactive reuse happens, we validated our
   approach with good speedups over traditional call subsumption.
   
\end{description}


\section{Further Work}

While we made some valuable contributions with this thesis, much work is left to be done in the future.
We next summarize further work that can and should be done:

\begin{description}
   
   \item[Experimentation in real world applications.] Our retroactive-based tabling engine still needs more
   experimentation and testing with real world data and applications in order to refine the implementation.
   More intensive experimentation may provide a deep analysis on the algorithms implemented and many opportunities
   to make each algorithm more smart will sure arise.
   
   \item[More efficient algorithms in the table space.] While the algorithms we used for the STST table space
   work pretty well for the majority of applications, there are certain programs were the overhead of various
   subgoals inserting on the same answer trie is considerable. Newer algorithms should be developed in order to,
   while preserving the good results for the majority of the cases, combat these deficiencies. In the other hand,
   the system does not benefit from a compile answer trie until the most general subgoal completes. Novel mechanisms
   must be developed in order to take advantage of the compiled tries optimization while inserting answers on the trie
   at the same time.
   
   \item[Exploration of other execution models.] When using retroactive-based tabling, there are some cases where the
   engine needs to call multiple subgoals of the same predicate in order to calculate the answers for the top subgoal.
   Sometimes, it would be advantageous to \emph{abstract} the running subgoals into a more general subgoal and then
   call it. After that, the top subgoal would use the answers generated by the more general subgoal. This mechanism
   called \emph{call abstraction} was proposed by Johnson et al.~\cite{Johnson-99} and is based on the idea that it
   may be useful for some programs to lose goal directness and generate the full set of answers and then select relevant
   answers from that set. Retroactive-based tabling already includes all the machinery necessary to do that, which makes
   it a good framework to further support call abstraction by devising various analysis techniques of call patterns.
   
\end{description}