\section{Evaluation Models and Reachability}

Consider the \texttt{path/2} predicate which computes the reachability between two nodes in a graph.
This predicate can be defined in six different ways, namely: \texttt{left\_first}, \texttt{left\_last},
\texttt{right\_first}, \texttt{right\_last}, \texttt{double\_first}, and \texttt{double\_last}.
The six versions are presented in Figure~\ref{fig:path_versions}.

\begin{figure}[ht]
\begin{Verbatim}
% left_first
path(X, Y) :- path(X, Z), edge(Z, Y).
path(X, Y) :- edge(Z, Y).

% left_last
path(X, Y) :- edge(X, Y).
path(X, Y) :- path(X, Z), edge(Z, Y).

% right_first
path(X, Y) :- edge(X, Z), path(Z, Y).
path(X, Y) :- edge(X, Y).

% right_last
path(X, Y) :- edge(X, Y).
path(X, Y) :- edge(X, Z), path(Z, Y).

% double_first
path(X, Y) :- path(X, Z), path(Z, Y).
path(X, Y) :- edge(X, Y).

% double_last
path(X, Y) :- edge(X, Y).
path(X, Y) :- path(X, Z), path(Z, Y).
\end{Verbatim}
\caption{Six versions of the \texttt{path/2} program.}
\label{fig:path_versions}
\end{figure}

Given the a database of facts about the connectivity of the graph, we may want to know the
execution modes, either variant, subsumptive or retroactive, and the \texttt{path/2} version
that are best suited to compute a specific problem. In the next list we discuss each problem
and the most efficient manner to solve it. For the benchmarks, we use the \texttt{grid} graph,
which is the most complex graph configuration we have available, and can easily generalize
the conclusions for the other types of graphs. The grid size is 32.

\begin{description}
   \item[all the pairs of connected nodes:] This is the problem of finding all solutions of the predicate.
   To solve this we use the query goal `\texttt{?-~path(X,Y)}'.
   
   Clearly, the best way to approach this problem is to use either the \texttt{left\_first} or
   the \texttt{left\_last} programs, because only one consumer is allocated. All the other
   versions will generate more than one consumer for the most interesting graphs.
   In terms of execution modes, the best to use is the variant-based model, as the allocated consumer
   is variant of the called subgoal, therefore, there is no need to use subsumption mechanisms.
   The results presented in Table~\ref{fig:result_path_xy} validate our claims.
   
   \begin{table}[ht]
\centering
\footnotesize{
  \begin{tabular}{c|ccc}
   \hline
    \hline
    \multicolumn{1}{c|}{\small{\textbf{Data}}} & \textbf{\small{Variant}} & \textbf{\small{Subsumptive}} & \textbf{\small{Retroactive}} \\
   \hline
   \hline

double\_first &  44282  &  \textbf{27661}  &  38858 \\
left\_first &  \underline{\textbf{216}}  &  29893  &  30149 \\
left\_last &  \textbf{220}  &  256  &  268 \\
right\_first &  312  &  \textbf{236}  &  252 \\
right\_last &  \textbf{324}  &  328  &  332 \\
\hline
\hline
\end{tabular}
}
\caption{Results for the query goal `\texttt{?-~path(X,Y)}'}
\label{tbl:results_path_xy}
\end{table}
   
   \item[the set of nodes that can be reached by starting in a specific node:] Starting in node $n1$, which
   nodes can be reached? In Prolog we write this as `\texttt{?-~path($n1$,X)}'.
   
   For this situation, the \texttt{left\_first} or \texttt{left\_last} will attain the best results,
   because, as for the first problem, only one consumer is stored. All the other versions, will create
   multiple consumers, which will slow down the computation. Like the first problem, the best execution
   model is variant-based tabling, because the consumer is variant of the subgoal \texttt{?-~path($n1$,X)}.
   
   Table~\ref{fig:result_path_1x} present the results for the query goal `\texttt{?-~path(1,X)}'.
   For this benchmark, all evaluation models performed equally good on the \texttt{left} programs.
   
   \begin{table}[ht]
\centering
\footnotesize{
  \begin{tabular}{c|ccc}
   \hline
    \hline
    \multicolumn{1}{c|}{\small{\textbf{Data}}} & \textbf{\small{Variant}} & \textbf{\small{Subsumptive}} & \textbf{\small{Retroactive}} \\
   \hline
   \hline

double\_first &  \textbf{17321}  &  24421  &  41010 \\
double\_last &  \textbf{17445}  &  23885  &  32266 \\
left\_first &  \textbf{4}  &  \textbf{4}  &  \textbf{4} \\
left\_last &  \textbf{4}  &  \textbf{4}  &  \textbf{4} \\
right\_first &  \textbf{140}  &  180  &  292 \\
right\_last &  \textbf{156}  &  176  &  304 \\
\hline
\hline
\end{tabular}
}
\caption{Results for the query goal `\texttt{?-~path(1,X)}'}
\label{tbl:results_path_1x}
\end{table}
   
   \item[the set of nodes from which a specific node can be reached:] Ending in node $n1$,
   we want to know the nodes that are connected to $n1$. In Prolog: `\texttt{?-~path(X,$n1$)}'.
   
   In variant-based tabling, for the \texttt{double\_first} and \texttt{double\_last} programs, we call the
   subgoal \texttt{path(X,Y)} as a generator, followed by multiple generators with the all arguments instantiated.
   If using subsumption-based tabling the subgoal \texttt{path(X,Y)} is called, followed by multiple consumers.
   For the \texttt{right\_first} and \texttt{right\_last} programs, with variant checks, we will create multiply
   generators with all the arguments instantiated or multiple generators using subsumptive checks.
   For the \texttt{left\_first} or \texttt{left\_last} programs, with either variant or subsumptive checks,
   the subgoal \texttt{path(X,Y)} will be called.
   
   Using retroactive tabling we can do better by using the \texttt{left\_first} program version.
   In this version, we will call a more subgoal, \texttt{path(X,Y)}, that will prune the subgoal
   \texttt{path(X,$n1$)}, and \texttt{path(X,$n1$)} will consume answers from \texttt{path(X,Y)}.
   Using the \texttt{left\_last} version is worse, because the subsumed subgoal will execute the first
   clause, and only then call the subsuming subgoal, thus diminishing what we will gain from pruning
   the execution.
   
   \item{table_path_x1}
   
   \item[if two nodes are connected:] Given two nodes $n1$ and $n2$ of the graph, we want to know if
   they are connected. In Prolog, we write this as: `\texttt{?-~path($n1$,$n2$)}'.
   
   Using variant-based tabling, the \texttt{right} or \texttt{double} versions, execution will generate multiple
   generators. For the \texttt{left} versions, evaluation creates a generator for the subgoal
   \texttt{path($n1$,Z)}, followed by a consumer of this subgoal.
   
   For subsumption-based tabling, both the \texttt{right} and \texttt{double} versions will create multiple
   generators, because subsumed subgoals are called before more general subgoals. For the \texttt{left} version,
   we have the same situation as for variant-based tabling.
   
   As for the previous problem, retroactive-based tabling is also best suited for this problem when using the
   \texttt{left\_first} program. Here, we call the subgoal \texttt{path($n1$,Z)} that will subsumed the query
   subgoal and prune it, thus saving time executing unnecessary code. Like the previous problem,
   the basic idea is to generate all the answers for a more general subgoal, and then use those answers to solve
   the initial problem. 
\end{description}