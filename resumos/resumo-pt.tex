%-----------------------------------------------
% Template para criação de resumos de projectos/dissertação
% jlopes AT fe.up.pt,   Fri Jul  3 11:08:59 2009
%-----------------------------------------------

\documentclass[9pt,a4paper]{extarticle}

%% English version: comment first, uncomment second
\usepackage[portuguese]{babel}  % Portuguese
%\usepackage[english]{babel}     % English
\usepackage{graphicx}           % images .png or .pdf w/ pdflatex OR .eps w/ latex
\usepackage{times}              % use Times type-1 fonts
\usepackage[utf8]{inputenc}     % 8 bits using UTF-8
\usepackage{url}                % URLs
\usepackage{multicol}           % twocolumn, etc
\usepackage{float}              % improve figures & tables floating
\usepackage[tableposition=top]{caption} % captions
%% English version: comment first (maybe)
\usepackage{indentfirst}        % portuguese standard for paragraphs
%\usepackage{parskip}

%% page layout
\usepackage[a4paper,margin=30mm,noheadfoot]{geometry}

%% space between columns
\columnsep 12mm

%% headers & footers
\pagestyle{empty}

%% figure & table caption
\captionsetup{figurename=Fig.,tablename=Tab.,labelsep=endash,font=bf,skip=.5\baselineskip}

%% heading
\makeatletter
\renewcommand*{\@seccntformat}[1]{%
  \csname the#1\endcsname.\quad
}
\makeatother

%% avoid widows and orphans
\clubpenalty=300
\widowpenalty=300

\begin{document}

\title{\vspace*{-8mm}\textbf{\textsc{Call Subsumption Mechanisms for Tabled Logic Programs}}}
\author{\emph{Flávio Manuel Fernandes Cruz}\\[2mm]
\small{Projecto/Dissertação realizado sob a orientação do \emph{Prof.\ Ricardo Rocha}}\\
\small{no \emph{CRACS / FCUP}}}
\date{}
\maketitle
%no page number 
\thispagestyle{empty}

\vspace*{-4mm}\noindent\rule{\textwidth}{0.4pt}\vspace*{4mm}

\begin{multicols}{2}

\section{Estado da Arte}

A tabulação é um método de resolução particularmente bem sucedido que resolve algumas das limitações
do método de avaliação SLD encontrado em sistemas Prolog \cite{Chen-96}, no tratamento de computações recursivas e/ou redundantes. Em comparação com o método SLD, a tabulação consegue reduzir o espaço de procura, evitar computações
redundantes e tem melhores propriedades de terminação \cite{Tamaki-86}.
Na tabulação, as primeiras chamadas a subgolos tabelados são avaliadas normalmente através da execução do código do
programa, enquanto que \emph{chamadas similares} são avaliadas através do consumo das respostas guardadas
na \emph{tabela} e que foram geradas pelo subgolo similar correspondente.
Em geral, podemos distinguir entre duas formas de determinar se um subgolo $A$ é similar a um subgolo $B$:
\emph{tabulação por variantes} e \emph{tabulação por subsumpção}.
Na tabulação por variantes, $A$ é similar a $B$ quando eles são iguais por renomeação das variáveis.
Na tabulação por subsumpção, $A$ é similar a $B$ quando $A$ é mais específico do que $B$ (ou $B$ é mais geral do que $A$).
Isto acontece pelo simples princípio de que se $A$ é mais específico do que $B$ e $S_A$ e $S_B$ são os respectivos
conjuntos de respostas, então $S_A \subseteq S_B$.
Embora a tabulação por subsumpção consiga atingir maiores ganhos em termos do tempo de execução, devido
à maior partilha de respostas, a implementação eficiente dos mecanismos necessários para seu suporte é bastante
mais difícil em comparação com tabulação por variantes, o que faz com que este último seja bastante mais popular
entre os motores de tabulação disponíveis.

\section{Objectivos}

Esta tese descreve a migração e integração do mecanismo de \emph{Time Stamped Tries} (TST) do motor de tabulação
SLG-WAM \cite{Sagonas-PhD} no motor de tabulação YapTab \cite{Rocha-00a}. Este mecanismo foi proposto
por Ernie Johnson \textit{et al.} \cite{Johnson-99,Johnson-00} e implementa os algoritmos e estruturas
de dados que suportam tabulação por subsumpção. A técnica TST é baseada na ideia de estender a tabela com
informação temporal para se distinguir respostas novas das antigas.

Na segunda parte desta tese apresenta-se o desenho, implementação e avaliação de uma nova extensão baseada na tabulação
por subsumpção chamada \emph{Tabulação por Subsumpção Retroactiva} (TSR) \cite{cruz-jelia10}. A TSR resolve algumas limitações da
tabulação por subsumpção tradicional, nomeadamente, o facto de a ordem da chamada dos subgolos poder afectar o seu sucesso e
aplicação. A TSR permite uma partilha completa e bidireccional de respostas entre subgolos, independentemente da
sua ordem de chamada através do corte da avaliação dos golos mais específicos.
Para implementar tabulação retroactiva foram desenvolvidas algumas novas ideias: (1) um
algoritmo eficiente para pesquisar, na tabela, subgolos mais específicos; (2) uma nova organização da tabela,
onde as respostas são representadas apenas uma vez; (3) uma nova estratégia de execução capaz de cortar
e transformar a execução de um subgolo gerador em consumidor.
   
\section{Resultados}

Os nossos resultados mostram que a integração dos mecanismos e algoritmos da TST da SLG-WAM para o YapTab
foram bem sucedidos, com desempenhos comparáveis aos da SLG-WAM na execução entre tabulação por variantes e
tabulação por subsumpção.

Para a TSR, os nossos resultados mostram ganhos consideráveis para os programas que conseguem tirar
partido do novo mecanismo, enquanto que o custo associado aos programas que dele não conseguem beneficiar
é quase insignificante.
   
\section{Conclusões}

As contribuições desta tese são as seguintes: (1) suporte para tabulação por subsumpção implementada no Yap Prolog;
(2) a nova técnica TSR que permite partilha bidireccional de respostas; (3) uma nova organização da tabela que
permite uma melhor partilha entre subgolos de um predicado; e (4) um motor de tabulação capaz de usar diferentes
técnicas de execução, tais como tabulação por variantes, por subsumpção e por subsumpção retroactiva. O nosso sistema final
permite que o programador escolha a melhor estratégia por predicado, o que torna a tabulação mais prática e
usável em problemas reais.

%%English version: comment first, uncomment second
\bibliographystyle{unsrt-pt}  % numeric, unsorted refs
%\bibliographystyle{unsrt}  % numeric, unsorted refs
\bibliography{refs}

\end{multicols}

\end{document}
