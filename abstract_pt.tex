A tabulação é um método de resolução particularmente bem sucedido que resolve algumas das limitações
do método de avaliação SLD encontrado em sistemas Prolog, no tratamento de computações recursivas e/ou redundantes.
Na tabulação, as primeiras chamadas a subgolos tabelados são avaliadas normalmente através da execução do código do
programa, enquanto que \emph{chamadas similares} são avaliadas através do consumo das respostas guardadas
na \emph{tabela} e que foram geradas pelo subgolo similar correspondente.
Em geral, podemos distinguir entre duas formas de determinar se um subgolo $A$ é similar a um subgolo $B$:
\emph{tabulação por variantes} e \emph{tabulação por subsumpção}.
Na tabulação por variantes, $A$ é similar a $B$ quando eles são iguais por renomeação das variáveis.
Na tabulação por subsumpção, $A$ é similar a $B$ quando $A$ é mais específico do que $B$ (ou $B$ é mais geral do que $A$).
Isto acontece pelo simples princípio de que se $A$ é mais específico do que $B$ e $S_A$ e $S_B$ são os respectivos
conjuntos de respostas, então $S_A \subseteq S_B$.
Embora a tabulação por subsumpção consiga atingir maiores ganhos em termos do tempo de execução, devido
à maior partilha de respostas, a implementação eficiente dos mecanismos necessários para seu suporte é bastante
mais difícil em comparação com tabulação por variantes, o que faz com que este último seja bastante mais popular
entre os motores de tabulação disponíveis.

Esta tese descreve a migração e integração do mecanismo de \emph{Time Stamped Tries} do motor de tabulação
SLG-WAM no motor de tabulação YapTab. Os resultados obtidos mostram que os nossos esforços de integração foram
bem sucedidos, com desempenhos comparáveis aos da SLG-WAM na execução entre tabulação por variantes e tabulação
por subsumpção.

Na segunda parte desta tese apresenta-se o desenho, implementação e avaliação de uma nova extensão baseada na tabulação
por subsumpção chamada \emph{Tabulação por Subsumpção Retroactiva} (TSR). A TSR resolve algumas limitações da
tabulação por subsumpção tradicional, nomeadamente, o facto de a ordem da chamada dos subgolos poder afectar o seu sucesso e
aplicação. A TSR permite uma partilha completa e bidireccional de respostas entre subgolos, independentemente da
sua ordem de chamada através do corte da avaliação dos golos mais específicos.
Os nossos resultados mostram ganhos consideráveis para os programas que conseguem tirar partido do novo mecanismo,
enquanto que o custo associado aos programas que dele não conseguem beneficiar é quase insignificante.
