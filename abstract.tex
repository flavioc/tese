Tabling is a particularly successful resolution mechanism that overcomes some limitations
of the SLD resolution method found in Prolog systems, namely, in dealing with recursion and redundant
sub-computations. In tabling, first calls to tabled subgoals are evaluated through
program resolution, while \emph{similar calls} are evaluated by consuming answers stored
in the table space by the corresponding similar subgoal.
In general, we can distinguish between two main approaches to determine if a subgoal $A$ is
similar to a subgoal $B$: \emph{variant-based tabling} and \emph{subsumption-based tabling}.
In variant-based tabling, $A$ is similar to $B$ if they are the same
up to variable renaming. In subsumption-based tabling, $A$ is similar to $B$ when $A$ is subsumed
by $B$ (or $B$ subsumes $A$). This stems from a simple principle: if $A$ is subsumed by $B$ and
$S_A$ and $S_B$ are the respective answer sets, then $S_A \subseteq S_B$.
While subsumption-based tabling (or \emph{call subsumption}) can yield superior time performance
by allowing greater answer reuse, its efficient implementation is harder than variant-based tabling,
which makes tabling engines with variant checks much more popular in the logic community.

This thesis first addresses the porting and integration of the \emph{Time Stamped Tries} mechanism
from the SLG-WAM tabling engine into the YapTab tabling engine. Our performance results show that our
integration efforts were successful, with comparable speedups when using subsumptive-based tabling against
variant-based tabling.

In the second part of this thesis we present the design, implementation, and evaluation of a novel extension
based on subsumption-based tabling called \emph{Retroactive Call Subsumption} (RCS). RCS overcomes some limitations
of traditional call subsumption, namely, the fact that the call order of the subgoals can greatly affect its
success and applicability. RCS allows full sharing of answers,
independently of the order they are called by selectively pruning and restarting the evaluation of subsumed
subgoals. Our results show considerable gains for programs that can take advantage of RCS, while programs
that do not benefit from it show a small overhead using the new mechanisms.
