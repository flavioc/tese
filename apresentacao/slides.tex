\documentclass{beamer}

\usepackage{beamerthemeBoadilla}
%\usepackage{beamerthemeCambridgeUS}
\usepackage{url}
\usepackage{verbatim}
\usepackage[portuguese]{babel}
\usepackage[utf8]{inputenc}
\usepackage{multirow}

\def\Tiny{\fontsize{6pt}{6pt}\selectfont}

\title{Call Subsumption Mechanisms for Tabled Logic Programs}
\author[Flávio Cruz]{Flávio Cruz {\small \texttt{<flaviocruz@gmail.com>}}\\
Orientador: Ricardo Rocha {\small \texttt{<ricroc@dcc.fc.up.pt>}}}
\institute[CRACS]
{
  \inst{1}%
  Center for Research in Advanced Computing Systems
  \and
  \vskip-2mm
  \inst{2}%
  Faculdade de Ciências da Universidade do Porto
}
\date{\today}

\begin{document}

\frame{\titlepage}

\section[Outline]{}
\frame{\tableofcontents}

\section{Prolog e o método SLD}{}

\frame
{
  \frametitle{Prolog e o método SLD}
  \begin{itemize}
     \item <+-> Na programação em lógica, o método de resolução SLD é um método inerentemente não-deterministico e do tipo \textit{top-down}.
     \item <+-> Em Prolog usa-se o método SLD de forma determinística, avaliando as cláusulas de cima para baixo e da esquerda para a direita.
     \item <+-> Esta forma de avaliação pode ser aplicado de forma eficiente em máquinas virtuais baseadas em stack, tais como a \emph{Warren's Abstract Machine} (WAM).
     \item <+-> No entanto, este método tem diversas limitações, tais como o tratamento de ciclos infinitos (positivos ou negativos) e computações redundantes.
     
  \end{itemize}
}

\subsection{Limitações}

\begin{frame}[fragile]
  \frametitle{Limitações do método SLD}
  \begin{columns}[t]
     \column{.5\textwidth}
     \begin{block}{Programa}
       {\small
       \begin{verbatim}
path(X, Z) :- path(X, Y),
             edge(Y, Z).
path(X, Z) :- edge(X, Z).

edge(1, 2).
edge(2, 3).
       \end{verbatim}
       }
       \begin{figure}[ht]
         \centering
           \includegraphics[scale=0.6]{edges.pdf}
       \end{figure}
     \pause
     \end{block}
      \column{.4\textwidth}
      \begin{block}{\texttt{?- path(1,~Z)}}
        \begin{figure}[ht]
          \centering
            \includegraphics[scale=0.8]{inf.pdf}
        \end{figure}
      \end{block}
      
  \end{columns}
\end{frame}

\section{Tabulação}

\begin{frame}
    \frametitle{Tabulação}
    \begin{itemize}
       \item <+-> A tabulação é um refinamento do método de resolução SLD.
       \item <+-> As primeiras chamadas a subgolos tabelados são avaliados normalmente através da execução do código do
       programa.
       \item <+-> As \emph{chamadas similares} são avaliadas através do consumo das respostas guardadas na \emph{tabela}
       e que foram geradas pelo subgolo similar correspondente.
       \item <+-> Permite que programas válidos em termos lógicos sejam executáveis.
    \end{itemize}
\end{frame}

\subsection{Similaridade de subgolos}

\begin{frame}
   \frametitle{Similiaridade entre chamadas}
   Em geral, existem dois testes para verificar se um subgolo é similar a outro:
   \begin{itemize}
      \item \emph{Tabulação por variantes:} $A$ é similar a $B$ quando eles são iguais por renomeação das variáveis.
      \pause
      \begin{example}
         $p(X,1,Y)$ e $p(Y,1,Z)$ são variantes porque ambas podem ser transformadas em $p(VAR_0,1,VAR_1)$
      \end{example}
      \pause
      \item A maioria dos motores de tabulação, incluindo o YapTab, apenas suportam este teste
   \end{itemize}
\end{frame}

\begin{frame}
   \frametitle{Similiaridade entre chamadas}
   \begin{itemize}
   \item \emph{Tabulação por subsumpção:} $A$ é similar a $B$ quando $A$ é mais específico do que $B$ (ou $B$ é mais geral do que $A$).
   \begin{example}
      $p(X,1,2)$ é mais específico do que $p(Y,1,Z)$ porque existe uma substituição $\{Y~=~X,~Z~=~2\}$
      que torna $p(X,1,2)$ uma \emph{instância} de $p(Y,1,Z)$.
   \end{example}
   \pause
   \begin{theorem}
      Se $A$ é mais específico do que $B$ e $S_A$ e $S_B$ são os respectivos conjuntos de respostas, então $S_A \subseteq S_B$.
   \end{theorem}
   \pause
   \item Só o XSB Prolog implementa este tipo de tabulação, primeiro usando uma técnica chamada \textit{Dynamic Threaded Sequential Automata} (DTSA) e mais recentemente usando a técnica de \textit{Time Stamped Tries} (TST).
\end{itemize}
\end{frame}

\subsection{Exemplo}

\begin{comment}

\begin{frame}[fragile]
   \frametitle{Exemplo por variantes}
   {\tiny
   \begin{columns}[t]
         \column{.35\textwidth}
         \begin{block}{Respostas}
           {%\tiny
           \begin{itemize}
             \item<4->\alert<4>{(4) \texttt{Z = 2}}
             \item<7->\alert<7>{(7) \texttt{Z = 3}}
           \end{itemize}
           }
         \end{block}
         \column{.55\textwidth}
         \begin{block}{Programa}
           {%\tiny
             \begin{verbatim}
path(X, Z) :- path(X, Y), edge(Y, Z).
path(X, Z) :- edge(X, Z).

edge(1, 2). edge(2, 3).
               \end{verbatim}
           }
         \end{block}
     \end{columns}
     }
   \begin{example}
      \begin{center}
      \includegraphics<1>[height=4.5cm]{tabled_evaluation1.pdf}%
      \includegraphics<2>[height=4.5cm]{tabled_evaluation2.pdf}%
      \includegraphics<3>[height=4.5cm]{tabled_evaluation3.pdf}%
      \includegraphics<4>[height=4.5cm]{tabled_evaluation4.pdf}%
      \includegraphics<5>[height=4.5cm]{tabled_evaluation5.pdf}%
      \includegraphics<6>[height=4.5cm]{tabled_evaluation6.pdf}%
      \includegraphics<7>[height=4.5cm]{tabled_evaluation7.pdf}%
      \includegraphics<8>[height=4.5cm]{tabled_evaluation8.pdf}%
      \includegraphics<9>[height=4.5cm]{tabled_evaluation9.pdf}%
      \includegraphics<10>[height=4.5cm]{tabled_evaluation10.pdf}%
      \includegraphics<11>[height=4.5cm]{tabled_evaluation11.pdf}%
   \end{center}
   \end{example}
\end{frame}
\end{comment}

\begin{frame}[fragile]
   \frametitle{Exemplo por subsumpção}
   {\tiny
   \begin{columns}[t]
         \column{.35\textwidth}
         \begin{block}{Respostas}
           {%\tiny
           \begin{itemize}
              \item<1->\texttt{path(X, Z)}: \onslide<3->{\alert<3>{(3) \texttt{X=1 Z=2}}}
                     \onslide<4->{\alert<4>{(4) \texttt{X=2 Z=3}}}
                     \onslide<8->{\alert<8>{(8) \texttt{X=1 Z=3}}}
              \item<7->\texttt{path(2, Z)}: \onslide<8->{\alert<8>{(8) \texttt{Z=3}}}
              \item<9->\texttt{path(3, Z)}: \onslide<11->{$\emptyset$}
           \end{itemize}
           }
         \end{block}
         \column{.55\textwidth}
         \begin{block}{Programa}
           {%\tiny
           \begin{verbatim}
path(X, Z) :- edge(X, Z).
path(X, Z) :- edge(X, Y), path(Y, Z).

edge(1, 2). edge(2, 3).
               \end{verbatim}
           }
         \end{block}
     \end{columns}
     }
   \begin{example}
      \begin{center}
      \includegraphics<1>[height=5.0cm]{sub1.pdf}%
      \includegraphics<2>[height=5cm]{sub2.pdf}%
      \includegraphics<3>[height=5cm]{sub3.pdf}%
      \includegraphics<4>[height=5cm]{sub4.pdf}%
      \includegraphics<5>[height=5cm]{sub5.pdf}%
      \includegraphics<6>[height=5cm]{sub6.pdf}%
      \includegraphics<7>[height=5cm]{sub7.pdf}%
      \includegraphics<8>[height=5cm]{sub8.pdf}%
      \includegraphics<9>[height=5cm]{sub9.pdf}%
      \includegraphics<10>[height=5cm]{sub10.pdf}%
      \includegraphics<11>[height=5.0cm]{sub11.pdf}%
   \end{center}
   \end{example}
\end{frame}

\begin{frame}
   \frametitle{Espaço das tabelas}
   Como são implementadas as tabelas?
   \begin{itemize}
      \item Tries: estruturas em árvore onde os prefixos comuns dos termos são representados apenas uma vez.
      \pause
      \item Normalmente, existem dois níveis de tries:
         \begin{itemize}
            \item \emph{Subgoal trie}: guarda os subgolos para um certo predicado (por exemplo \texttt{path/2}).
            \item \emph{Answer trie}: guardas as respostas.
         \end{itemize}
      \pause
      \item Num nó folha da subgoal trie existe uma estrutura chamada \emph{subgoal frame} que contém informação sobre o subgolo respectivo.
   \end{itemize}
\end{frame}

\begin{frame}
  \frametitle{Espaço das tabelas}
  \begin{example}
  \begin{figure}[ht]
     \centering
       \includegraphics[scale=0.6]{two_level_tries.pdf}
   \end{figure}
 \end{example}
\end{frame}

\section{Time Stamped tries}

\begin{frame}
  \frametitle{Time Stamped Tries}
  \begin{itemize}
     \item Estende-se a answer trie com informação temporal: \emph{timestamps}.
     \item Quando uma resposta é inserida, incrementa-se o timestamp da resposta.
     \item O objectivo é permitir uma pesquisa incremental de respostas para os subgolos mais específicos.
     \item O subgolo específico guarda o timestamp da última procura para evitar respostas repetidas no futuro.
     \item O algoritmo de pesquisa faz corte dos ramos pelo timestamp e através de operações de unificação ao longo da trie.
  \end{itemize}
\end{frame}

\begin{frame}
  \frametitle{Time Stamped Tries}
  \begin{itemize}
    \item Time stamped trie do subgolo \texttt{p(X, Y, Z)}:
  \end{itemize}
  \begin{columns}[t]
      \column{.4\textwidth}
      \begin{example}[antes]
        \begin{figure}[ht]
          \centering
            \includegraphics[scale=0.4]{tst_1.pdf}
        \end{figure}
      \end{example}
      \column{.5\textwidth}
      \begin{example}[inserir \textbf{p(a, b, c)}]
        \begin{figure}[ht]
          \centering
            \includegraphics[scale=0.4]{tst_2.pdf}
        \end{figure}
      \end{example}
  \end{columns}
\end{frame}

\subsection{Implementação}

\begin{frame}
   \frametitle{Implementação no YapTab}
   \begin{itemize}
      \item Reutilizou-se o código de forma quase integral do XSB Prolog: foram usados macros para
      permitir que ambos os sistemas Prolog usassem o mesmo código.
      \item As alterações nas operações principais de tabulação foram mínimas.
      \begin{itemize}
         \pause
         \item Cálculo do líder.
         \pause
         \item Nova chamada.
         \pause
         \item Nova resposta.
         \pause
         \item Calcular a próxima resposta a consumir.
         \pause
      \end{itemize}
      \item Todas as instruções da trie tiveram que ser alteradas para usar unificação por forma a serem usadas em subgolos específicos quando completas.
      \item O sistema permite usar uma mistura de predicados por variantes e por subsumpção.
   \end{itemize}
\end{frame}

\section{Tabulação por Subsumpção Retroactiva}

\subsection{Motivação}

\begin{frame}
   \frametitle{Problemas na tabulação por subsumpção}
   \begin{itemize}
      \item Apesar da tabulação por subsumpção atingir bons resultados, sofre de um problema:
      a ordem na qual os subgolos são chamados pode afectar a performance do sistema.
      
      \begin{example}
         Se \texttt{p(1,X)} for chamado antes de \texttt{p(X,Y)}, \texttt{p(1,X)} não usará as respostas de
         \texttt{p(X,Y)}, mas irá executar o código para gerar as suas próprias respostas.
      \end{example}
      
      \item Assim, para existir partilha de respostas entre subgolos subsumptivos é estritamente necessário
      que o golo mais específico apareça depois do subgolo mais geral.
   \end{itemize}
\end{frame}

\subsection{Tabulação por Subsumpção Retroactiva}

\begin{frame}
   \frametitle{Como solucionar este problema?}
   \begin{itemize}
      \item Tabulação por Subsumpção Retroactiva (TSR).
      \pause
      \item Quando um subgolo $G$ é chamado, cortam-se os ramos de execução do subgolo mais
      específico $G'$ para transformar $G'$ num consumidor.
      \item Assim, $G'$ passa a usar as soluções de $G$ e deixa de gerar as suas próprias soluções.
      \item O corte de ramos de execução potencia ganhos de tempo de execução e a partilha de respostas ganhos
      em termos de utilização de memória.
      \pause
      \item Desafios:
      \begin{itemize}
         \item Determinar que subgolos geradores e consumidores pertencem à execução do subgolo $G'$.
         \begin{itemize}
            \pause
            \item Construindo uma árvore das dependências dos subgolos.
         \end{itemize}
         \pause
         \item Manter a execução consistente devido aos cortes.
         \begin{itemize}
            \pause
            \item Considerando intervalos de pontos de escolha.
         \end{itemize}
      \end{itemize}
   \end{itemize}
\end{frame}

\begin{frame}[fragile]
   \frametitle{Exemplo de TSR}
   \begin{columns}[]
   \column{.35\textwidth}
   \begin{block}{Programa}
     {\tiny
     \begin{verbatim}
:- use_retroactive_tabling p/2.

a(X) :- p(1, X).

p(1, 3). p(2, 3). p(1, 2).
      \end{verbatim}
     }
   \end{block}
   \column{0.55\textwidth}
   {\small
   \begin{block}{TSR}
      \only<1>{Subgolo \texttt{p(X,Y)} é mais geral que \texttt{p(1,X)}}
      \only<2>{Subgolo \texttt{p(X,Y)} torna-se num nó retroactivo}
      \only<3>{Dado que o subgolo mais geral completou, o nó retroactivo transforma-se num nó \emph{loader}
      e carrega as soluções relevantes}
   \end{block}
   }
\end{columns}
   \begin{example}
      \begin{center}
      \includegraphics<1>[height=4.5cm]{retro1.pdf}%
      \includegraphics<2>[height=4.5cm]{retro2.pdf}%
      \includegraphics<3>[height=4.5cm]{retro3.pdf}%
   \end{center}
   \end{example}
\end{frame}

\subsection{Corte da execução}

\begin{frame}
   \frametitle{Corte da execução}
   \begin{itemize}
      \item Existem dois tipos de corte dependendo onde o subgolo mais geral aparece relativamente ao subgolo específico:
      \begin{itemize}
         \item Corte externo: se aparece fora.
         \item Corte interno: se aparece dentro.
      \end{itemize}
      \pause
      \item Independente do tipo de corte, existe um conjunto de problemas que advém do corte da execução
      de geradores ou consumidores internos ao subgolo específico:
      \begin{itemize}
         \item \emph{Orphaned Consumers}
         \item \emph{Lost consumers}
         \item \emph{Pseudo-Completion}
         \item \emph{Leader Re-Computation}
      \end{itemize}
      \pause
      \item Após o corte, o ponto de escolha do subgolo específico é transformado num nó retroactivo,
      para que possa haver \emph{resolução retroactiva}.
      \begin{itemize}
         \item A instrução que implementa resolução retroactiva é vital para a TSR, pois permite que
         os nós de execução sejam transformados em tipos de nós correctos e desta forma, permitir
         que a computação termine.
      \end{itemize}
   \end{itemize}
\end{frame}

\setbeamerfont{block}{size=\tiny}

\begin{frame}[fragile]
   \frametitle{Corte Externo}
   \begin{columns}[]
   \column{.35\textwidth}
   \begin{block}{Programa}
     {\Tiny
     \begin{verbatim}
:- use_variant_tabling [a/2, b/1].
:- use_retroactive_tabling p/2.

a(X, Y) :- p(1, X), b(Y).
a(3, 4).
b(1). b(2).
p(1, X) :- a(_, X).
p(1, X) :- b(X).
      \end{verbatim}
     }
   \end{block}
   \column{0.55\textwidth}
   {\small
   \begin{block}{TSR}
      \only<1>{Novo gerador \texttt{a(X,Y)}}
      \only<2>{Novo gerador \texttt{p(1,X)}}
      \only<3>{Novo consumidor \texttt{a(\_,X)}}
      \only<4>{Novo gerador \texttt{b(X)}}
      \only<5>{Novo consumidor \texttt{b(Y)}}
      \only<6>{Novo gerador mais geral \texttt{p(Z,W)}}
      \only<7>{Determinar ramos a cortar}
      \only<8>{Consumidores como o \texttt{a(\_,X)} são simplesmente removidos do \emph{dependency space}}
      \only<9>{\texttt{b(X)} é um subgolo gerador interno, mudar o seu estado para \emph{pruned}.
      Transformar \emph{consumidores externos} (\emph{orphaned consumers}) em nós retroactivos}
      \only<10>{O nó \texttt{b(Y)} é um \emph{nó fronteira}. É necessário ligá-lo ao nó \texttt{p(1,X)} para
      evitar que a execução salte para ramos mortos}
      \only<11>{Transformar o nó \texttt{p(1,X)} em nó retroactivo e remover o \emph{subgoal frame} da
      pilha respectiva}
      \only<12>{Novo consumidor \texttt{a(\_,W)}}
      \only<13>{Gerador \texttt{b(V0)} é reactivado e completa}
      \only<14>{Subgolo \texttt{p(Z,W)} tenta completar mas não é líder}
      \only<15>{Após backtracking, o nó retroactivo \texttt{b(Y)} executa a instrução de resolução retroactiva
      e transforma-se num nó carregador (\emph{loader})}
      \only<16>{Ao chegar-mos ao nó \texttt{p(1,X)}, este é transformado num consumidor, pois \texttt{p(Z,W)}
      ainda não completou}
      \only<17>{O subgolo \texttt{a(X,Y)} como líder poderá depois completar a computação em segurança}
   \end{block}
   }
\end{columns}
   \begin{example}
      \begin{center}
         \includegraphics<1>[height=4.1cm]{issues1.pdf}%
         \includegraphics<2>[height=4.1cm]{issues2.pdf}%
         \includegraphics<3>[height=4.1cm]{issues3.pdf}%
         \includegraphics<4>[height=4.1cm]{issues4.pdf}%
         \includegraphics<5>[height=4.1cm]{issues5.pdf}%
         \includegraphics<6>[height=4.1cm]{issues6.pdf}%
         \includegraphics<7>[height=4.1cm]{issues7.pdf}%
         \includegraphics<8>[height=4.1cm]{issues8.pdf}%
         \includegraphics<9>[height=4.1cm]{issues9.pdf}
         \includegraphics<10>[height=4.1cm]{issues10.pdf}
         \includegraphics<11>[height=4.1cm]{issues11.pdf}
         \includegraphics<12>[height=4.1cm]{issues12.pdf}
         \includegraphics<13>[height=4.1cm]{issues13.pdf}
         \includegraphics<14>[height=4.1cm]{issues14.pdf}
         \includegraphics<15>[height=4.1cm]{issues15.pdf}
         \includegraphics<16>[height=4.1cm]{issues16.pdf}
         \includegraphics<17>[height=4.1cm]{issues17.pdf}
      \end{center}
   \end{example}
\end{frame}

\begin{frame}
   \frametitle{Corte Interno}
   \begin{itemize}
      \item O corte interno acontece quando o subgolo mais geral $G$ aparece dentro da execução do subgolo específico $G'$.
      \item Nesta situação cortam-se os ramos referentes a $G'$, excepto a parte que irá
      computar as soluções do subgolo $G$.
      \pause
      \item $G$ executa normalmente mas não devolve as soluções para o ambiente externo \footnote{O subgolo
      executa usando \emph{local scheduling}}.
      \item Quando $G$ ou completar ou não conseguir completar por ser o líder, salta-se para o ponto
      de escolha do subgolo $G'$, onde este poderá carregar as suas soluções relevantes ou transformar-se em
      consumidor.
      \pause
      \item O nosso sistema é capaz de detectar situações de cortes internos múltiplos.
   \end{itemize}
\end{frame}

\subsection{Espaço das tabelas}
\begin{frame}
   \frametitle{Espaço das tabelas}
   \begin{itemize}
      \item \emph{Single Time Stamped Trie}
      \item Uma \emph{answer trie} única por predicado.
      \pause
      \item As respostas são representas apenas uma vez e referenciadas pelos subgolos que as usam.
      \item Usa-se um timestamp por cada subgolo de forma a facilitar a transformação de gerador para consumidor.
      \begin{itemize}
         \item Situações em que diferentes subgolos estão a inserir respostas na trie requerem cuidado.
      \end{itemize}
      \pause
      \item Permite que se possam reutilizar respostas relevantes a um novo subgolo gerador antes de executar
      o código.
      \begin{itemize}
         \item Esta é uma forma elegante de reutilizar respostas e resolver o problema das tabelas incompletas.
      \end{itemize}
      \pause
      \item Tem como desvantagem a necessidade de representar todos os argumentos de um dado subgolo e não
      apenas o valor das variáveis.
   \end{itemize}
\end{frame}

\subsection{Procura de subgolos específicos}
\begin{frame}
   \frametitle{Procura de subgolos específicos}
   \begin{itemize}
      \item Uma componente importante da TSR é o algoritmo que percorre a \emph{subgoal trie} para
      encontrar subgolos mais específicos.
      \pause
      \item O problema resume-se a encontrar atribuições para as variáveis do subgolo mais geral.
      \pause
      \item A pesquisa é feita navegando pela trie e usando backtracking sempre que a pesquisa falhar.
      \item Usa-se uma pilha de nós alternativos de pesquisa.
   \end{itemize}
\end{frame}

\begin{frame}
   \frametitle{Procura de subgolos específicos}
   \begin{itemize}
      \item Dado que é necessário construir termos Prolog e registar atribuições de variáveis durante a pesquisa,
      usa-se estruturas da máquina virtual, tais como a heap e a trilha.
      \pause
      \item Para melhorar a eficiência, estendeu-se cada nó da \emph{subgoal trie} com um campo chamado
      \texttt{in\_eval} que registo o número de subgolos sobre aquele ramo da trie que estão a executar.
   \end{itemize}
   \pause
   \begin{columns}[t]
        \column{.45\textwidth}
        \begin{block}{Subgoal trie}
          \begin{figure}[ht]
            \centering
              \includegraphics[scale=0.33]{in_eval_trie.pdf}
          \end{figure}
        \pause
        \end{block}
         \column{.45\textwidth}
         \begin{block}{Novo gerador}
           \begin{figure}[ht]
             \centering
               \includegraphics[scale=0.33]{in_eval_add.pdf}
           \end{figure}
         \end{block}

     \end{columns}
\end{frame}

\section{Resultados}

\begin{frame}
   \frametitle{Tabulação por subsumpção}
   \begin{itemize}
      \item Avaliou-se o desempenho do motor de tabulação por subsumpção e comparou-se com
      o motor de tabulação do XSB Prolog.
      \item Sendo que ambos usam os mesmos algoritmos e estruturas de dados, o desempenho é muito
      parecido.
   \end{itemize}
   \pause
   {\footnotesize
   \begin{center}
     \begin{tabular}{ccc}
      \hline
       \hline
       \multirow{2}{*}{\textbf{Programa}} & \textbf{XSB Prolog} & \textbf{Yap Prolog} \\
       & \textit{\small{Speedup médio}} & \textit{\small{Speedup médio}} \\
      \hline
      \hline
   left\_first & 0.78 & \textbf{1.02} \\
   left\_last & 0.77  & \textbf{0.96} \\
   right\_first & \textbf{1.01} & \textbf{1.01} \\
   right\_last & 0.94 & \textbf{1.07} \\
   double\_first & 1.37 & \textbf{1.48} \\
   double\_last & 1.31 & \textbf{1.40} \\
   samegen & \textbf{339.76} & 1.03 \\
   genome & 559.54 & \textbf{648.51} \\
   reach\_first  & \textbf{0.96} & 0.94 \\
   reach\_last  & \textbf{0.97} & 0.90 \\
   \hline
   \hline
   \end{tabular}
   \end{center}}
\end{frame}

\begin{frame}
   \frametitle{Custo da TSR}
   \begin{itemize}
      \item Foi medido o desempenho da TSR para programas que não tiram vantagens de usar os novos
      mecanismos.
      \item Comparou-se o desempenho com tabulação por subsumpção tradicional e tabulação por variantes.
   \end{itemize}
   \pause
   \begin{center}
      {\footnotesize
     \begin{tabular}{ccc}
      \hline
       \hline
       \multirow{2}{*}{\textbf{Programa}} & \multicolumn{2}{c}{\textbf{Yap Prolog}} \\
       & \textit{\small{Retro / Var}} & \textit{\small{Retro / Sub}} \\
      \hline
      \hline
      left\_first & 1.06 & 1.01 \\
      left\_last &  1.07  & 1.03 \\
      right\_first & \textbf{0.97} & \textbf{0.95} \\
      right\_last & 1.25 & \textbf{0.94} \\
      double\_first & 1.01 & 1.16 \\
      double\_last & 1.04 & 1.16 \\
      samegen & 1.19 & 1.14 \\
      reach\_first  &  1.11  & 1.04 \\
      reach\_last  &  1.17  & 1.04 \\
   \hline
   \hline
   \textit{Média Total} &  1.10 &  1.05 \\
   \hline
   \hline
   \end{tabular}}
\end{center}
\end{frame}

\begin{frame}
   \frametitle{Ganhos da TSR}
   
   \begin{itemize}
      \item Por outro lado, comparou-se o desempenho para programas onde usar TSR é vantajoso.
   \end{itemize}
   \pause
   \begin{center}
   {\footnotesize
     \begin{tabular}{ccc}
      \hline
       \hline
       \multirow{2}{*}{\textbf{Programa}} & \multicolumn{2}{c}{\textbf{Yap Prolog}} \\
       & \textit{\small{Var / Retro}} & \textit{\small{Sub / Retro}} \\
      \hline
      \hline
   left\_first & 0.89 & 0.95 \\
   left\_last & 0.88  & 0.90 \\
   double\_first & 1.07 & \textbf{1.09} \\
   double\_last & 1.05 & \textbf{1.10} \\
   genome & 450.33 & 0.74 \\
   reach\_first  & 2.54 & \textbf{1.76} \\
   reach\_last  & 3.22 & \textbf{1.87} \\
   flora & 3.17 & \textbf{1.17} \\
   fib & 1.95 & \textbf{2.02} \\
   big & 13.26 & \textbf{13.66} \\
   \hline
   \hline
   \end{tabular}}
   \end{center}
   \begin{itemize}
      \item Para os programas do tipo \texttt{path/2} usou-se o golo \texttt{path(X,1)}.
   \end{itemize}
\end{frame}

\section{Conclusões}

\begin{frame}
   \frametitle{Conclusões}
   \begin{itemize}
      \item Contribuições:
      \begin{itemize}
         \item YapTab suporta tabulação por subsumpção.
         \item Mecanismos e algoritmos que controlam a execução retroactiva.
         \item Algoritmo de pesquisa de subgolos específicos.
         \item Espaço de tabelas inovador que permite uma maior reutilização de respostas.
         \item Suporte para uma mistura de métodos de avaliação: retroactivo, variantes e subsumpção.
         \pause
      \end{itemize}
      \item Trabalho futuro:
      \begin{itemize}
         \item Integrar o trabalho na distribuição oficial do Yap Prolog.
         \item Melhoramento dos algoritmos do espaço das tabelas.
         \item Maior experimentação com aplicações reais.
         \item Explorar outros métodos de avaliação, como o \emph{Call Abstraction}.
         \pause
      \end{itemize}
   \end{itemize}
\end{frame}

\begin{frame}
   \frametitle{Artigos Publicados}
   \begin{itemize}
      \item Retroactive Subsumption-Based Tabled Evaluation of Logic Programs, Flávio Cruz and Ricardo Rocha. 12th European Conference on Logics in Artificial Intelligence (JELIA 2010), Springer-Verlag. Helsinki, Finland, September 2010.
      \item Submetidos:
         \begin{itemize}
      \item Efficient Instance Retrieval of Executing Subgoals for Tabled Evaluation, Flávio and Ricardo Rocha. 17th International Conference on Logic for Programming, Artificial Intelligence and Reasoning (LPAR 17), Yogyakarta, Indonesia, October 2010.
      \item Efficient Retrieval of Subsumed Subgoals in Tabled Logic Programs, Flávio Cruz and Ricardo Rocha. Compilers, Programming Languages, Related Technologies and Applications (CORTA 2010), Braga, Portugal, September 2010.
   \end{itemize}
   \end{itemize}
\end{frame}
   
\end{document}