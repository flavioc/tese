This chapter introduces the concept of retroactive call subsumption (RCS). RCS
enables full sharing of answers among subgoal calls, independent of the order they are called,
using the relation of subsumption.

First, we introduce the motivations behind RCS by illustrating the shortcomings of traditional call
subsumption mechanisms. Next, we present the concepts introduced by RCS and how execution
rules are extended to support retroactive tabling. Other extensions are then discussed, namely:
the new table space organization based around the ideas of the \textit{common global trie} proposal
\cite{CostaJ-08} and the algorithm to traverse the call trie to search for subsumed subgoals. Finally, we
give some details about the implementation of this new extension.

\section{Motivation}

In traditional call subsumption, a new call to the subgoal $G$ is considered to create a producer subgoal
when a more general subgoal $G'$ is not found on the call trie and it is the first time $G$ is called.
When $G'$ exists, a new consumer is stored to consume answers from the subsuming subgoal $G'$, thus creating
a consumer subgoal.
