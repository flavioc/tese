%-----------------------------------------------
% Template para criação de resumos de projectos/dissertação
% jlopes AT fe.up.pt,   Fri Jul  3 11:08:59 2009
%-----------------------------------------------

\documentclass[9pt,a4paper]{extarticle}

%% English version: comment first, uncomment second
\usepackage[portuguese]{babel}  % Portuguese
%\usepackage[english]{babel}     % English
\usepackage{graphicx}           % images .png or .pdf w/ pdflatex OR .eps w/ latex
\usepackage{times}              % use Times type-1 fonts
\usepackage[utf8]{inputenc}     % 8 bits using UTF-8
\usepackage{url}                % URLs
\usepackage{multicol}           % twocolumn, etc
\usepackage{float}              % improve figures & tables floating
\usepackage[tableposition=top]{caption} % captions
%% English version: comment first (maybe)
\usepackage{indentfirst}        % portuguese standard for paragraphs
%\usepackage{parskip}

%% page layout
\usepackage[a4paper,margin=30mm,noheadfoot]{geometry}

%% space between columns
\columnsep 12mm

%% headers & footers
\pagestyle{empty}

%% figure & table caption
\captionsetup{figurename=Fig.,tablename=Tab.,labelsep=endash,font=bf,skip=.5\baselineskip}

%% heading
\makeatletter
\renewcommand*{\@seccntformat}[1]{%
  \csname the#1\endcsname.\quad
}
\makeatother

%% avoid widows and orphans
\clubpenalty=300
\widowpenalty=300

\begin{document}

\title{\vspace*{-8mm}\textbf{\textsc{Call Subsumption Mechanisms for Tabled Logic Programs}}}
\author{\emph{Flávio Manuel Fernandes Cruz}\\[2mm]
\small{Projecto/Dissertação realizado sob a orientação do \emph{Prof.\ Ricardo Rocha}}\\
\small{no \emph{CRACS / FCUP}}}
\date{}
\maketitle
%no page number 
\thispagestyle{empty}

\vspace*{-4mm}\noindent\rule{\textwidth}{0.4pt}\vspace*{4mm}

\begin{multicols}{2}

\section{Background}

Tabling is a particularly successful resolution mechanism that overcomes some limitations
of the SLD resolution method found in Prolog systems \cite{Chen-96}, namely, in dealing with
recursion and redundant sub-computations. In comparison to the traditional SLD resolution method, tabling can reduce the search space to cut redundant computations,
avoids looping and has better termination properties \cite{Tamaki-86}.
In tabling, first calls to tabled subgoals are evaluated
through program resolution, while \emph{similar calls} are evaluated by consuming answers stored
in the table space by the corresponding similar subgoal.
In general, we can distinguish between two main approaches to determine if a subgoal $A$ is
similar to a subgoal $B$: \emph{variant-based tabling} and \emph{subsumption-based tabling}.
In variant-based tabling, $A$ is similar to $B$ if they are the same
up to variable renaming. In subsumption-based tabling, $A$ is similar to $B$ when $A$ is subsumed
by $B$ (or $B$ subsumes $A$). This stems from a simple principle: if $A$ is subsumed by $B$ and
$S_A$ and $S_B$ are the respective answer sets, then $S_A \subseteq S_B$.
While subsumption-based tabling (or \emph{call subsumption}) can yield superior time performance
by allowing greater answer reuse, its efficient implementation is harder than variant-based tabling,
which makes tabling engines with variant checks much more popular in the logic community.

\section{Purpose}

This thesis first addresses the porting and integration of the \emph{Time Stamped Tries} (TST)
mechanism from the SLG-WAM \cite{Sagonas-PhD} into YapTab \cite{Rocha-00a}.  This mechanism was proposed by
Ernie Johnson \textit{et al}. \cite{Johnson-99,Johnson-00} and implements the algorithms and data structures
that support subsumption-based tabling. The TST technique is based on the idea of extending the table space with
time information to distinguish between new answers from old answers.

In the second part of this thesis we present the design, implementation, and evaluation of a novel extension
based on subsumption-based tabling called \emph{Retroactive Call Subsumption} (RCS) \cite{cruz-jelia10}.
RCS overcomes some limitations of traditional call subsumption, namely, the fact that the call order of the subgoals
can greatly affect its success and applicability. RCS allows full sharing of answers,
independently of the order they are called by selectively pruning and restarting the evaluation of
subsumed subgoals. To implement retroactive-based tabling we developed a few novel ideas: (1) a novel algorithm
to efficiently retrieve running \emph{instances} of a subgoal; (2) a novel table
space organization, where answers are represented only once; and (3) a new evaluation strategy capable
of pruning and transforming generator subgoals into consumer subgoals.
   
\section{Results}
   
Our performance results show that the integration of TST mechanisms and algorithms from the SLG-WAM
to YapTab was largely successful, with comparable speedups when using subsumptive-based tabling
against variant-based tabling.

For the RCS engine, our results show that the overhead of the new mechanisms for RCS support are
low enough in programs that do not benefit from it, which, combined with considerable gains for
programs that can take advantage of them, validates this new evaluation technique.
   
\section{Conclusions}

The main contributions of this thesis are the following: (1) support for subsumption-based tabling in Yap Prolog;
(2) the novel RCS technique that permits bidirectional reuse of answers; (3) a novel table space
organization that enhances answer reuse by allowing answer reuse on a predicate basis; and 
(4) a tabling engine capable of mixing multiple evaluating strategies, such
as variant, subsumption and retroactive-based tabling. Our final system enables the programmer to
choose the best evaluation strategy per predicate, which arguably can augment the power of tabling
for real world programming.

%%English version: comment first, uncomment second
%\bibliographystyle{unsrt-pt}  % numeric, unsorted refs
\bibliographystyle{unsrt}  % numeric, unsorted refs
\bibliography{refs}

\end{multicols}

\end{document}
