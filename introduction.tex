Logic programming is a very high level programming paradigm that allows the programmer
to focus on the declarative aspects of the problem, instead of describing the specific steps
needed to solve the problem. Arguably, the Prolog language is the most popular logic
programming language. Part of the success of the Prolog language can be attribute to the
development of a fast and very efficient sequential machine called the \emph{Warren Abstract Machine}
(WAM) \cite{Warren-83}. The advances in WAM technology and optimization techniques enabled Prolog
to be applied in real world problems in a wide range of fields such as Artificial Intelligence,
Natural Language Processing, Machine Learning, Knowledge Based Systems, Database Management, or
Expert Systems.

While the declarative aspect of Prolog is based on mathematical logic and the predicate calculus,
the Prolog language employs a relatively simple refutation strategy called \emph{Selective Linear Definite}
(SLD) \cite{Lloyd-87}, which is a well defined evaluation method for logic programs that
is particularly well suited to stack based machines.
Furthermore, Prolog defines a few extra-logical constructs, such as the \emph{cut operator}
that give the programmer more control over the evaluation. Both the SLD operational semantics
and the cut operator make the programmer more aware of the actual evaluation process in detriment to the
declarative aspect of the language that is naturally non-deterministic. It is possible to exploit
the deterministic rules and extra-logical features of Prolog to speedup the execution of its execution,
thus, for example, solve problems related to redundant sub-computations. Notwithstanding, standard Prolog has
still some deficiencies. For instance, writing left-recursive programs can lead to infinite loops.

There have been some attempts in making Prolog less prone to problems related to recursion
and redundant sub-computations, in order to make Prolog more expressive and closer to mathematical logic.
One of these attempts, which is particularly successful, is called \emph{tabling}
(or \emph{tabulation} or \emph{memoization} \cite{Michie-68}). The tabling technique stems from one simple idea:
store intermediate answers in a place called the \emph{table space} and reuse those answers when a
\emph{similar call} appears during the resolution process. Tabling refines the SLD resolution method
by distinguishing between first calls to \emph{tabled subgoals}, which are evaluated as usual through
\emph{program resolution}, and similar calls to tabled subgoals that are evaluated through \emph{answer resolution}
by consuming answers that are being stored in the table space by the corresponding similar subgoal, instead
of being re-evaluated against the program clauses. Tabled evaluation is able to reduce the search space,
avoid looping, and has better termination properties than traditional SLD resolution \cite{Chen-96}.
The advantages of tabling have lead to its application in fields such as Deductive Databases \cite{Sagonas-94},
Program Analysis \cite{RamakrishnanCR-00}, Knowledge Based Systems \cite{Yang-00}, Inductive Logic
Programming \cite{Rocha-05b}, and Model Checking~\cite{RamakrishnanCR-00}.

In tabling, \emph{call similarity} (or \emph{checks}) determines if a subgoal $A$ is similar to subgoal $B$,
in other words, wether $A$ will generate its own answers or if it will consume from $B$. In general,
we can distinguish between two main approaches for call similarity:

\begin{itemize}
   \item \emph{Variant-based tabling}: $A$ and $B$ are variants if they can be made identical
   through variable renaming as proposed by Bachmair \textit{et al} \cite{Bachmair-93}.
   For example, subgoals $p(X,1,Y)$ and $p(Y,1,Z)$ are \emph{variants},
   because both can be transformed into $p(VAR_0,1,VAR_1)$;
   \item \emph{Subsumption-based tabling} or \emph{tabling by call subsumption}: Subgoal $A$ is considered similar
   to $B$ if $A$ is \emph{subsumed} by $B$ (or $B$ \emph{subsumes} $A$) if $A$ is more specific than $B$
   (or an instance of). For example, subgoal $p(X,1,2)$ is subsumed by subgoal $p(Y,1,Z)$ because there
   is a substitution $\{Y~=~X,~Z~=~2\}$ that makes $p(X,1,2)$ an instance of $p(Y,1,Z)$. Tabling by call
   subsumption is based on the principle that if $A$ is subsumed by $B$ and $S_A$ and $S_B$ are the respective
   answer sets, therefore $S_A \subseteq S_B$.
\end{itemize}

In general, subsumption-based tabling has the following advantages over variant-tabling:
superior time performance, because less program resolution is required; and less space requirements,
as it allows greater reuse of answers, since the answer sets for the subsumed subgoals are not stored.
However, the mechanisms to efficiently support subsumption-based tabling are more complex and harder to
implement, which makes the variant-based tabling approach more popular within the available tabling systems,
such as YapTab \cite{Rocha-00a}, B-Prolog \cite{Zhou-00}, and ALS-Prolog \cite{Guo-01}.
To the best of our knowledge, the SLG-WAM \cite{Sagonas-98} engine from XSB Prolog is the sole tabling system that supports
subsumption-based tabling, initially by using an organization of the table space called
\emph{Dynamic Threaded Sequential Automata (DTSA)}~\cite{Rao-96}, and later by using an alternative design called
\emph{Time-Stamped Tries (TST)}~\cite{Johnson-99},.
While the DTSA design was more efficient, it had bigger space requirements and thus is was replaced by the simpler
TST approach which uses far less memory.

\section{Thesis Purpose}

In this thesis we address the design, implementation, integration and evaluation of two subsumption-based engines
built on top of the YapTab tabling system. For the first system, we reused and integrated the Time-Stamped Trie
approach from XSB Prolog into Yap Prolog. We attempted to reuse both code and data structures and then
we integrated these new mechanisms into YapTab. We studied how subsumption-based and variant-based tabling were seamlessly
integrated into the SLG-WAM engine. Consequently, we made minimal modifications to the YapTab engine
that enabled it to support a mix of variant and subsumptive subgoals on the same program.
Our performance results show that our integration efforts were successful, with comparable
speedups when using subsumptive-tabling against variant-tabling.

For the second system, we designed a novel extension for subsumptive-tabling called
\emph{Retroactive Call Subsumption (RCS)}.
This extension attempts to solve one major problem in traditional call subsumption: the order in
which subgoals are called during a particular evaluation can greatly affect the success and applicability
of the call by subsumption technique. For example, if more specific subgoals are called before
the more general subgoal, no reuse will be employed, while if the more general subgoal is called first,
reuse will happen. The RCS extends the original TST design by allowing full sharing of answers, independently
of the order they are called. The basic idea is to selectively prune and restart the evaluation of generator
subgoals that are subsumed by a new called subgoal in order to reuse the answers from the subsuming subgoal,
instead of continuing to generate their own answers.

To implement retroactive-based tabling we developed a few novel ideas: (1) a novel algorithm to efficiently
traverse and retrieve running \emph{instances} of a subgoal; (2) a novel table space organization, based on
the ideas of the \emph{common global trie} proposal~\cite{CostaJ-08}, where answers are represented only
once; and (3) a new evaluation strategy capable of pruning and transforming generator nodes into consumer nodes.

Our results show that the overhead of the new control mechanisms of RCS are low enough in programs that do not
benefit from it, which, combined with considerable gains for programs that can take advantage of them, validates
this new evaluation technique. In this in mind, we argue that Retroactive Call Subsumption makes tabling
more adapted and useful for practical applications and is another great tool in the programmer's toolbox for
writing tabled programs.

\section{Thesis Outline}

In the following list we describe each chapter of this thesis.

\begin{description}

   \item[Chapter 1: Introduction.] Is this chapter.
   
   \item[Chapter 2: Logic Programming and Tabling.] Provides a overview of the main topics of this thesis.
   The subjects discussed are logic programming, Prolog, and tabling for logic programs. A brief description
   of the YapTab and SLG-WAM tabling engines is also presented.
   
   \item[Chapter 3: Table Space Organization.] Describes the table space organization for both variant and
   subsumption-based tabling engines. We start by describing the variant table space for both SLG-WAM and
   YapTab systems. We then give a brief overview about the table space organization for the DTSA and TST
   techniques that implement tabling with subsumptive checks.
   
   \item[Chapter 4: Time Stamped Tries.] Throughly presents the Time Stamped Tries approach to subsumption-based
   tabling. First, we describe the algorithm used to detect subsuming subgoals on the subgoal trie. Next,
   we give a detailed description of the data structures used in the table space that are used to speedup
   the identification of relevant answers for subsumed subgoals. Finally, we focus on the modifications we have
   made in the YapTab tabling engine in order to support tabling by call subsumption.
   
   \item[Chapter 5: Retroactive Call Subsumption.] We start with the motivations behind RCS, by showing the shortcomings
   of pure subsumption-based tabling. We next describe the rules for the new mechanism and the problems that arise
   when pruning execution branches. Finally, we discuss the novel table space called \emph{Single Time Stamped Trie}
   and then we throughly describe the new algorithm we developed to find executing subsumed subgoals of a subgoal on the
   table space.
   
   \item[Chapter 6: Experimental Results.] This chapter first presents the experimental results we achieve with
   the new YapTab engine that reuses the TST approach and how it compares to the SLG-WAM. We also present and
   discuss the overhead of the RCS mechanism on programs that do not benefit from it and the speedups we have
   achieve for programs that can take advantage of it.
   
   \item[Chapter 7: Conclusions.] Summarizes the work, enumerates the contributions and suggests directions for
   further work.
   
\end{description}
