This chapter explores the concept of retroactive call subsumption (RCS). RCS
enables full sharing of answers among subgoal calls, independent of the order they are called,
using the relation of subsumption.

First, we introduce the motivations behind RCS by illustrating the shortcomings of traditional call
subsumption mechanisms. Next, we present the concepts introduced by RCS and how execution
rules are extended to support retroactive tabling. Other extensions are then discussed, namely:
the new table space organization based around the ideas of the \textit{common global trie} proposal
\cite{CostaJ-08} and the algorithm to traverse the call trie to search for subsumed subgoals. Finally, we
give some details about the implementation of this new extension.

\section{Motivation}

In traditional call subsumption, a new call to the subgoal $G$ is considered to create a producer subgoal
when a more general subgoal $G'$ is not found on the call trie and it is the first time $G$ is called.
When $G'$ exists, a new consumer is stored to consume answers from the subsuming subgoal $G'$, thus creating
a consumer subgoal.

Consider that subgoal \texttt{p(X,1,2)} is called first, followed by \texttt{p(X,1,Z)}.
Notice that when \texttt{p(X,1,2)} is called it is considered a producer subgoal as no subsuming subgoals
are found on the call trie. The subgoal \texttt{p(X,1,Z)} is also considered as a producer, because
\texttt{p(X,1,2)} does not subsume \texttt{p(X,1,Z)}. If the call order is swapped, \texttt{p(X,1,Z)} continues
to be considered a producer subgoal, but now \texttt{p(X,1,2)} finds \texttt{p(X,1,Z)} as a subsuming subgoal
on the call trie, and thus is considered a consumer subgoal.

While call subsumption provides good results in terms of memory utilization and time it suffers from a
major problem: the order in which the the subgoals are called can greatly affect the performance
and applicability of the technique. Therefore, we introduce a new mechanism called \textit{retroactive call
subsumption} (RCS) that solves this problem by retroactively changing past tabled nodes to enable full sharing
of answers between subsuming and subsumed subgoals, independently of the order they are called.